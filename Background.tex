\section{Background}
\label{sec:Background}
%\vspace{0.10in}
% Use PNG or PDF file formats and no spaces
IoTs offer a promising future, and the idea of smart cities has grown impressively with the ascent and improvement of the IoT advancement. Smart cities are distributed systems that enable a diverse set of services over multiple layers of communication. Due to these characteristics a general architecture to fit all sizes is challenging. Andrea Zanella et al. \cite{zanella_internet_2014} present a framework that targets smart city street light application to collect environmental data (e.g. CO level, temperature, humidity, etc.) through wireless nodes placed on street lights to ensure the correct operation of the public lighting system. The study results demonstrate that the current technologies have achieved the dimension of development to permit the pragmatic acknowledgment of IoT solutions. 

In another study, Yibin Li et al. \cite{li_privacy_2016} address the privacy concern by presenting a mobile-cloud framework to protect user-data from smartphones and a proof-of-concept for the proposed benchmark to prevent data over-collection. The study shows the level of security risks in non-iOS devices is more frequent then iOS devices due to the locked development environment for iOS applications. The study also identifies the permission model of current mobile phones which is limited to allow full permission or none. The proposed benchmark overcomes this limitation and provides a customized permission model to address the data over-collection problem. P.G.V Naranjo et al. \cite{naranjo_focan:_2018} propose FOCAN, a Fog-supported multi-tire smart city network architecture that supports communications between heterogeneous systems through a smart city environment. The study supports Fog computing and aims to focus on scalability, energy saving, and low latency by introducing additional Fog layer in a distributed smart city environment. 

SpeedyChain is a new Blockchain-based framework~\cite{michelin_speedychain:_2018} that provides a private and secure communication model for smart vehicles. It also ensures data integrity using hashes in transactions as well as the communication auditability using transaction records in the Blockchain. The proposed framework decouples the block header and the transaction data to enable faster data addition to the block. The experiment results show a linear growth in time, showing low latency introduced by the proposed framework and the possibility to exchange information in smart cities ensuring resilience, data integrity and tamper-resistance.

On one hand, it is essential to understand the strengths and weaknesses of the Blockchain technology before implementation. Authors in \cite{dinh_untangling_2018} conducted a comprehensive study between three different Blockchain applications (Ethereum, Parity and Hyperledger) in terms of their data processing capabilities with the help of a proposed benchmark called ``BLOCKBENCH". The benchmark narrows the design space to get focused insights on the design trade-offs and performance bottlenecks. On the other hand, it is equally important to understand the concept of smart cities. Hafedh Chourabi et al. \cite{chourabi_understanding_2012} proposed a framework to understand smart cities where they have identified critical factors of smart city initiatives. Authors in \cite{liangsmartcity} analyze potential security concerns while integrating Blockchain in smart city infrastructure for enahnced resilience.

In all of the related studies, none of the researchers have emphasized the actual node deployment for a smart city infrastructure. This paper aims to provide adaptability in characterizing and deploying the smart city node framework, empowering a city IT administrator to customize the network size with any number of nodes. In the next section, we will discuss the deployment methodology for an experimental smart city architecture with different deployment scenarios.  

\section{Introduction}
\label{sec:introduction}
%\vspace{0.10in}
Internet-of-Things (IoT) \cite{dorri_Blockchain_2017} devices are influencing our daily lives in many ways by collecting and sharing data from the operating environment with the help of attached sensors. IoTs are gaining popularity and are widely used nowadays in smart home applications, medical and healthcare, transportation, manufacturing, supply chain \cite{liangEDS}, battlefield \cite{toshIOBT} and electric grids. In the case of a smart home, many connected devices receive/transmit data among themselves and to the outside world. The same example when expanded on a broader scale, a ``smart city''\cite{chourabi_understanding_2012}, where many different types of devices and technologies are integrated to monitor city operations will generate a vast amount of data. Thus the security and trustworthiness of each data point are critical for achieving efficiency in smart city applications. The concept of a smart city is still emerging, and some of the applications include smart parking, traffic congestion, smart grids \cite{ghoshsmartgrid2016}\cite{ghosh2017security}\cite{ghosh2019security}, etc. 

With the advancements of IoT and cyber technologies, the cities can be transformed to establish a connected, inter-operable, and highly automated infrastructure that will assist in taking effective decisions in real-time while achieving the necessary safety and security requirements of the society. However, instantiating such a huge cyber-infrastructure for the smart city applications can be overwhelming unless a systematic and well-analyzed deployment strategy is in place. There can be several unique challenges while deploying the infrastructure in a seamless manner, which can be categorized as follows:
(1) nature of the Infrastructure (complicated, costly, and technologically challenged);
(2) type of deployment (centralized, decentralized, distributed);
(3) scale of deployment (deployment, powering, and data collection from the devices);
(4) security of the infrastructure (authentication, validation, and access protection); 
(5) privacy (data confidentiality);
(6) inter-operability (interfacing heterogeneous systems).

Furthermore, the IoT devices usually are not computationally robust devices, and typically have limited processing power to execute core application functionalities. At the same time, there can be millions of sensors and other connections in a centralized fashion which introduces additional overhead in achieving data security and resiliency while communicating with the cloud server. Authors in  \cite{alaba_internet_2017} present an extensive study and a taxonomy on IoT security threats and vulnerabilities. The study discusses an IoT security scenario and proposes a client-server model. However, the proposed security model is not able to achieve the ideal secured environment due to the limitations of the IoT environment. Blockchain \cite{toshsecurity2017}, on the other hand, it provides a secure decentralized framework that eliminates single points of failure and offers critical security and resilient properties through its tamper-resistant distributed ledger. To achieve secure and resilient cyberinfrastructure in a smart city, the Blockchain technology can be considered as a potential candidate since the critical element of Blockchain, and the reason for its tremendous potential is the unchanging nature of its blocks. Blockchain offers the likelihood to have all the data in a single database with members having predefined authorizations to view or change the data they need. However, the difficulty involved in deploying Blockchain nodes one at a time requires an automated technique to build the infrastructure in a non-intrusive way.

In this paper, our contributions are the following: 1) propose a decentralized architecture for a smart city cyber infrastructure, 2) provide a non-intrusive Blockchain infrastructure implementation strategy to accommodate distributed nature and security needs of IoT-enabled smart city. Since it is essential to automate the bootstrapping process without redundant work in setting up a smart city infrastructure, we investigate the issue of automating the bootstrap process for Blockchain peers in a high node dynamism environment. To realize this scenario, we implement a simulated smart city environment (Fig. \ref{fig:Main_Architecture}) where different edge servers mimic the role of city departments that are deployed to collect and process data from connected devices and distribute them across other departments. Using a permissioned Blockchain, Hyperledger Sawtooth \cite{hyperledger-sawtooth}, and our proposed automation module, we demonstrate the seamless deployment of Blockchain infrastructure for the smart city use case. The proposed automation methodology results 82\% of deployment efficiency compared to traditional Blockchain deployment, thus significantly reduces the sequential deployment overheads.

The paper is organized as follows. In Section \ref{sec:Background}, we highlight the background and related work. Section \ref{sec:Architecture} discusses the proposed architecture and different deployment scenarios. In section \ref{sec:Evaluation}, we evaluate the results and discuss the observations. Section \ref{sec:Conclusion} discusses the findings and future directions. 


